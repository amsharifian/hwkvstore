\subsection{Pre-existing Components}
    Our system is built on a Xilinx ZC706 FPGA Evaluation platform. This board
    contains a Xilinx Zynq-7000 XC7Z045-2FFG900C AP SoC, with two ARM Cortex-A9
    MPCore application processors attached to programmable logic. Our full-system
    design runs on the programmable logic, with the ARM cores used only for 
    bootstrapping and non-network I/O purposes. In order to create an ethernet interface accessible
    to the programmable logic, we take advantage of the ZC706's SFP cage with a 
    Brocade 1Gb Copper SFP Transceiver.


    As our application processor, we use the RISC-V Rocket Core, an open-source
    64-bit, single-issue, in-order CPU running at 50MHz on the FPGA fabric. This 
    core is supported by Rocket-Chip, which contains uncore components such as
    caches and coherence agents, along with the host-machine interface (HTIF).
    In our system, the HTIF bus supports block devices, a console, and the ability
    to interact with the Rocket Core's DRAM.

    On our boards, Rocket has 512MiB of DRAM

Todo: Linux / poky



\subsection{Custom Components}
    Our original system built from pre-existing components was the standard
    distribution of the RISC-V Rocket Core for Xilinx Zynq FPGAs. As of the 
    start of our project, this distribution contained only a Rocket Core on
    the FPGA fabric, without I/O peripherals. As a result, we built many 
    components from scratch that would traditionally be available. We hope that
    the presence of these components on the open-source RISC-V platform
    will accelerate future research efforts. The following sections describe our
    experiences bringing up various hardware and software components that support
    our accelerator.

\subsubsection {SFP Bringup}
    In order to give the programmable logic access to a network, we use 
    the small form-factor pluggable (SFP) cage on the ZC706 board to house a 
    one gigabit SFP transceiver. Although the ZC706 contains an existing Ethernet
    PHY, it is tied to the ARM cores on the Zynq-7000 SoC, and attaching it to 
    the programmable logic is not feasible without introducing significant 
    latency overheads.

    The SFP requires a low-jitter 125MHz clock for operation. This is provided
    by an on-board Si5324 jitter attenuator chip. This chip does not default to 
    125MHz and instead must be programmed over I2C. Since the I2C bus is exposed
    only to the ARM cores on the Zynq SoC, bringup for this clock is the 
    responsibilty of a Linux driver running on the ARM core.

\subsubsection{Building the Network Interface Card}

    Our system uses Xilinx IP to construct a 1Gbps Network Interface Card

    \cite{pcspma}
    \cite{trimac}

\subsection{Bootstrapping the System}
    \subsubsection{ARM Core}

    When power is first supplied to the board, our full-system design, containing
    Rocket along with the traffic manager, accelerator, DMA engine, and NIC is 
    pushed onto the programmable logic in the Zynq SoC. Next, a copy of Linux
    with support for SFP clock bringup is booted on the ARM cores in the Zynq
    SoC. The \texttt{fesvr-zynq} application then executes on the ARM core. 
    This program copies a specified RISC-V binary into the Rocket Core's DRAM 
    over the HTIF bus and resets the Rocket Core. In our case, we supply a 
    \texttt{vmlinux} binary containing the RISC-V Linux kernel, along with a 
    root filesystem.

    \subsubsection{Rocket Core}

    From this point forward, the Rocket Core executes independently of the ARM 
    core. The ARM core now handles only non-network I/O for the Rocket core over
    the HTIF bus. 



\subsection{Future Infrastructure Work}
    Although our system functions correctly, we do not adhere to the RISC-V
    I/O specifications. (TODO: maybe this section is unnecessary)
